%%%%%%%%%%%%%%%%%%%%%%%%%%%%%%%%%%%%%%%%%%
% Dokument
%%%%%%%%%%%%%%%%%%%%%%%%%%%%%%%%%%%%%%%%%%
\documentclass[11pt,twoside]{scrartcl} %oneside Change xxxside -->marginpar must bee changed

\usepackage[pdftitle={\titleinfo},%
						pdfauthor={\authorinfo},%
						pdfcreator={pdfLatex, LaTeX with hyperref},
						pdfsubject={\subjectinfo},
						plainpages=false,
						pdfpagelabels,
						colorlinks,
						linkcolor=black,
						filecolor=black,
						citecolor=black,
						urlcolor=black]{hyperref}
						

% Headings
\usepackage{fancyhdr}
\renewcommand{\headrulewidth}{0.4pt}
\renewcommand{\footrulewidth}{0.4pt}	

\lhead[\scriptsize\nouppercase{\leftmark}]{\textbf{\titleinfo}}
\chead[V\versioninfo]{V\versioninfo}
\rhead[\textbf{\titleinfo}]{\scriptsize\nouppercase{\leftmark}}

\lfoot[\thepage]{\authorinfo}
\cfoot[]{}
\rfoot[\authorinfo]{\thepage}


%%%%%%%%%%%%%%%%%%%%%%%%%%%%%%%%%%%%%%%%%%
% Package's
%%%%%%%%%%%%%%%%%%%%%%%%%%%%%%%%%%%%%%%%%%
\usepackage{ucs}
\usepackage[utf8x]{inputenc}
\usepackage[T1]{fontenc}

\usepackage{layout}
\setlength{\parindent}{0em}

\usepackage{lscape}

\renewcommand{\baselinestretch}{1.2}
\renewcommand{\arraystretch}{1}

%Damit \today ein Deutsch Formatiertes Datum zurueckgibt.
\usepackage[ngerman, num, orig]{isodate}
\usepackage[german, ngerman]{babel}
\monthyearsepgerman{\,}{\,}

\usepackage{amssymb,amsmath,fancybox,graphicx,wrapfig,color,lastpage,verbatim,epstopdf,a4wide,tabularx, pdftricks}
\usepackage[usenames,dvipsnames]{pstricks}
\usepackage{setspace}
\usepackage{epsfig}
\usepackage{pst-pdf}
\usepackage{pst-all}
\usepackage{pstricks-add}
\usepackage{supertabular}
\usepackage[font=small,labelfont=bf]{caption}
\usepackage[font=footnotesize]{subfig}
\usepackage{footnote}
\usepackage{float}
\usepackage{multirow}
\usepackage{etex}
\usepackage{pdfpages}
\usepackage{pgf,tikz}
\usepackage{color}

\usepackage[makeroom]{cancel}
\usepackage{array}
\usepackage{trfsigns}
\usepackage{textcomp}
\usepackage{alltt} % code enviroment with math


\renewcommand{\captionfont}{\scriptsize\slshape}

%Neue Symbole für itemize
\newcommand{\cditem}[2]{\item[$\blacktriangleright$] \textbf{#1} \\ #2}
\newcommand{\cdrawitem}[1]{\item[$\blacktriangleright$] \textbf{#1}}
\newcommand{\cdsubitem}[2]{\item[$\vartriangleright$] \textbf{#1} \\ #2}

\newcommand{\fullquote}[2]{
	\begin{verse}
	 	\centering\emph{#1}
	\end{verse}
	
	\begin{flushright}
		$-$ \emph{#2}
	\end{flushright}
}
	
\setlength{\unitlength}{1mm}

%Inhaltsverzeichnis
\setcounter{secnumdepth}{3}
\setcounter{tocdepth}{3}

%Geometrie
\usepackage[a4paper,left=10mm,right=10mm,top=10mm,bottom=10mm,includeheadfoot]{geometry}

%%%%%%%%%%%%%%%%%%%%%%%%%%%%%%%%%%%%%%%%%%
% Randnotizen 
%%%%%%%%%%%%%%%%%%%%%%%%%%%%%%%%%%%%%%%%%%

%\setlength{\marginparwidth}{20mm} %Legt die Breite des Randnotizen-Bereichs fest.
%\setlength{\marginparpush}{10mm} %Legt den minimalen Abstand zwischen zwei Randnotizen fest.
%\setlength{\marginparsep}{3mm} %Legt den Abstand zwischen Text und Randnotizen fest.

%Info: Twosided
%\let\oldmarginpar\marginpar
%\renewcommand{\marginpar}[1]{
%	\ifthenelse{\isodd{\thepage}}
%	{ %then clause (gerade seiten)
%		\reversemarginpar
%		\vspace{\baselineskip}
%		\oldmarginpar[\raggedleft\footnotesize\textbf{#1}]{\raggedleft\footnotesize\textbf{#1}}
%		\normalmarginpar
%		\vspace{\baselineskip}
%		\oldmarginpar[\raggedleft\footnotesize\textbf{#1}]{\raggedleft\footnotesize\textbf{#1}}
%		\vspace{-\baselineskip}
%	}
%}

%Info: Onesided
%\let\oldmarginpar\marginpar
%\renewcommand{\marginpar}[1]{
%	\reversemarginpar
%	\vspace{\baselineskip}
%	\oldmarginpar[\raggedleft\footnotesize\textbf{#1}]{\raggedleft\footnotesize\textbf{#1}}
%	\vspace{-\baselineskip}
%}



%%%%%%%%%%%%%%%%%%%%%%%%%%%%%%%%%%%%%%%%%%
% Code Listings
%%%%%%%%%%%%%%%%%%%%%%%%%%%%%%%%%%%%%%%%%%
\usepackage{listings}
\definecolor{listinggray}{gray}{0.9}
\definecolor{lbcolor}{rgb}{0.95,0.95,0.95}
\definecolor{comment}{RGB}{34, 139, 34}
\lstset{
	backgroundcolor=\color{lbcolor},
	tabsize=4,
	rulecolor=,
	numbers=left,
	basicstyle=\scriptsize,
	upquote=true,
	aboveskip={1.0\baselineskip},
	columns=fixed,
	showstringspaces=false,
	extendedchars=true,
	breaklines=true,
	prebreak = \raisebox{0ex}[0ex][0ex]{\ensuremath{\hookleftarrow}},
	frame=single,
	showtabs=false,
	showspaces=false,
	showstringspaces=false,
	identifierstyle=\ttfamily,
	keywordstyle=\color[rgb]{0,0,1},
	commentstyle=\color{comment},
	stringstyle=\color{brown}, %\color[rgb]{0.627,0.126,0.941},
	captionpos=b,
}

\lstset{
	language=C++,
	directivestyle=\color{brown},	
}

\lstdefinelanguage[STL]{C++} [ANSI]{C++}{
	morekeywords=[2]{string, vector, list, map, std},
	%morecomment=[s]{{<}{>}},
	%commentstyle=\color{comment}
}

\lstdefinelanguage[Qt]{C++} [ANSI]{C++}{
	morekeywords=[2]{slot, signal, emit, foreach},
	morekeywords=[3]{slot, signal, emit, foreach}.
}

\lstdefinelanguage{cmd}{
	morecomment=[l]{\#},
	commentstyle=\color{comment}
}



%%%%%%%%%%%%%%%%%%%%%%%%%%%%%%%%%%%%%%%%%%%%%%%%%%%%%%%%%%%%%%%%
% Referenzen
%%%%%%%%%%%%%%%%%%%%%%%%%%%%%%%%%%%%%%%%%%%%%%%%%%%%%%%%%%%%%%%%

\newcommand{\figref}[1]{Abb.~\ref{#1}}
\newcommand{\subfigref}[2]{\figref{#1}.#2}
\renewcommand{\eqref}[1]{Gl.~\ref{#1}}
\newcommand{\tabref}[1]{Tabelle~\ref{#1}}
\renewcommand{\pageref}[1]{Seite~\ref{#1}}
\newcommand{\chapref}[1]{Kapitel~\ref{#1}}
\newcommand{\secref}[1]{Abschnitt~\ref{#1}}
\newcommand{\lstref}[1]{Listing~\ref{#1}}



%%%%%%%%%%%%%%%%%%%%%%%%%%%%%%%%%%%%%%%%%%%%%%%%%%%%%%%%%%%%%%%%
% Environment Numbering
%%%%%%%%%%%%%%%%%%%%%%%%%%%%%%%%%%%%%%%%%%%%%%%%%%%%%%%%%%%%%%%%

%Abbildungsnumerierung anhand Kapitel
\renewcommand{\thefigure}{\arabic{section}.\arabic{figure}}
\makeatletter \@addtoreset{figure}{section} \makeatother

%Gleichungen anhand Kapitel
\AtBeginDocument{\numberwithin{equation}{section}}
\AtBeginDocument{\numberwithin{figure}{section}}
\AtBeginDocument{\numberwithin{lstlisting}{section}}
\AtBeginDocument{\numberwithin{table}{section}}



%%%%%%%%%%%%%%%%%%%%%%%%%%%%%%%%%%%%%%%%%%%%%%%%%%%%%%%%%%%%%%%%
% Mathe
%%%%%%%%%%%%%%%%%%%%%%%%%%%%%%%%%%%%%%%%%%%%%%%%%%%%%%%%%%%%%%%%
\DeclareMathOperator{\sinc}{sinc}

\newcommand{\numbercircled}[1]{\textcircled{\raisebox{-1pt}{#1}}}

% Fouriertransformationen
\unitlength1cm
\newcommand{\FT}
{
	\begin{picture}(1,0.5)
	\put(0.2,0.1){\circle{0.14}}\put(0.27,0.1){\line(1,0){0.5}}\put(0.77,0.1){\circle*{0.14}}
	\end{picture}
}

% Inverse- Fouriertransformation
\newcommand{\IFT}
{
	\begin{picture}(1,0.5)
	\put(0.2,0.1){\circle*{0.14}}\put(0.27,0.1){\line(1,0){0.45}}\put(0.77,0.1){\circle{0.14}}
	\end{picture}
}



%%%%%%%%%%%%%%%%%%%%%%%%%%%%%%%%%%%%%%%%%%%%%%%%%%%%%%%%%%%%%%%%
% Farben
%%%%%%%%%%%%%%%%%%%%%%%%%%%%%%%%%%%%%%%%%%%%%%%%%%%%%%%%%%%%%%%%

%Farben
\definecolor{black}{rgb}{0,0,0}
\definecolor{red}{rgb}{1,0,0}
\definecolor{white}{rgb}{1,1,1}
\definecolor{grey}{rgb}{0.8,0.8,0.8}

%Markierungen
\newcommand{\draftmarker}[1]{\colorbox{yellow}{#1}}



%%%%%%%%%%%%%%%%%%%%%%%%%%%%%%%%%%%%%%%%%%%%%%%%%%%%%%%%%%%%%%%%
% Tabellen
%%%%%%%%%%%%%%%%%%%%%%%%%%%%%%%%%%%%%%%%%%%%%%%%%%%%%%%%%%%%%%%%

%\renewcommand{\arraystretch}{1.5} %Zeilenh�he von Tabellen

\newcolumntype{L}[1]{>{\raggedright\arraybackslash}p{#1}} % linksbündig mit Breitenangabe
\newcolumntype{C}[1]{>{\centering\arraybackslash}p{#1}} % zentriert mit Breitenangabe
\newcolumntype{R}[1]{>{\raggedleft\arraybackslash}p{#1}} % rechtsbündig mit Breitenangabe

\newcommand{\ltab}{\raggedright\arraybackslash} % Tabellenabschnitt linksb�ndig
\newcommand{\ctab}{\centering\arraybackslash} % Tabellenabschnitt zentriert
\newcommand{\rtab}{\raggedleft\arraybackslash} % Tabellenabschnitt rechtsb�ndig




%%%%%%%%%%%%%%%%%%%%%%%%%%%%%%%%%%%%%%%%%%%%%%%%%%%%%%%%%%%%%%%%
% Einheiten
%%%%%%%%%%%%%%%%%%%%%%%%%%%%%%%%%%%%%%%%%%%%%%%%%%%%%%%%%%%%%%%%
\usepackage[Gray,squaren]{SIunits} %\gray befehl heisst nun \Gray und \square heisst nun \squaren

%Spannung
\DeclareMathOperator{\V}{\volt}
\DeclareMathOperator{\mV}{\milli \volt}
\DeclareMathOperator{\uV}{\micro \volt}

%Strom
\DeclareMathOperator{\A}{\ampere}
\DeclareMathOperator{\mA}{\milli \ampere}
\DeclareMathOperator{\uA}{\micro \ampere}
\DeclareMathOperator{\nA}{\nano \ampere}

%Zeit
\DeclareMathOperator{\s}{\second}
\DeclareMathOperator{\ms}{\milli \second}
\DeclareMathOperator{\us}{\micro \second}
\DeclareMathOperator{\ns}{\nano \second}

%Kapazit�t
\DeclareMathOperator{\mF}{\milli \farad}
\DeclareMathOperator{\uF}{\micro \farad}
\DeclareMathOperator{\nF}{\nano \farad}
\DeclareMathOperator{\pF}{\pico \farad}
\DeclareMathOperator{\fF}{\femto \farad}

%Induktivit�t
\DeclareMathOperator{\mH}{\milli \henry}
\DeclareMathOperator{\uH}{\micro \henry}
\DeclareMathOperator{\nH}{\nano \henry}

%Widerstand
\DeclareMathOperator{\MO}{\mega \ohm}
\DeclareMathOperator{\kO}{\kilo \ohm}
\DeclareMathOperator{\mO}{\milli \ohm}
\DeclareMathOperator{\Ohm}{\ohm}
%Strecke
\DeclareMathOperator{\km}{\kilo \meter}
\DeclareMathOperator{\cm}{\centi \meter}
\DeclareMathOperator{\mm}{\milli \meter}

%Frequenz
\DeclareMathOperator{\GHz}{\giga \hertz}
\DeclareMathOperator{\MHz}{\mega \hertz}
\DeclareMathOperator{\Hz}{\hertz}
\DeclareMathOperator{\kHz}{\kilo \hertz}
\DeclareMathOperator{\mHz}{\milli \hertz}

%Leistung
\DeclareMathOperator{\kW}{\kilo \watt}
\DeclareMathOperator{\mW}{\milli \watt}
\DeclareMathOperator{\uW}{\micro \watt}
\DeclareMathOperator{\W}{\watt}

%Kreisfrequenz
\DeclareMathOperator{\rpers}{\radianpersecond}

%DeziBel
\DeclareMathOperator{\dB}{\deci \bel}
\DeclareMathOperator{\dBm}{\deci \bel \milli}

%Bit
\DeclareMathOperator{\Bit}{\text{Bit}}
\DeclareMathOperator{\kBit}{\text{kBit}}
\DeclareMathOperator{\MBit}{\text{MBit}}
\DeclareMathOperator{\Byte}{\text{Byte}}
\DeclareMathOperator{\kByte}{\text{kByte}}
\DeclareMathOperator{\MByte}{\text{MByte}}
\DeclareMathOperator{\ppm}{\text{ppm}}



%%%%%%%%%%%%%%%%%%%%%%%%%%%%%%%%%%%%%%%%%%%%%%%%%%%%%%%%%%%%%%%%
% Kommandos
%%%%%%%%%%%%%%%%%%%%%%%%%%%%%%%%%%%%%%%%%%%%%%%%%%%%%%%%%%%%%%%%
\newcommand{\todo}[1]{\colorbox{red}{#1}}
\newcommand{\baeni}[1]{$_{\textcolor{red}{\mbox{\small{Bäni S. #1}}}}$}