\section{Designing Wavelet Filters}

\subsection{Vanishing Moments}
 Wenn die Fouriertransformation $\hat{\psi}(\xi)$ eine Nullstelle der Ordnung $p$ an der Stelle $\xi = 0$ hat, dann sind die Ableitungen $\hat{\psi}^{(k)}(0)$ für $k=0,1,...,p-1$ gleich Null. Setzt man die Formel für die Fouriertransformation ein sieht man, dass das k-th Moment verschwindet! Anders ausgedrückt, verschwinden die Wavelet-Koeffizienten $v_{m,n}$ wenn die Funktion $f(t)$ über dem Träger des Wavelets mit einem Polynom des Grades $p-1$ dargestellt werden kann. 

\[ 
	\mu_k = \int_{-\infty}^{\infty} \psi(t) \cdot t^k \, \mathrm{d}t
\]

Ein Wavelet sollte möglichst viel verschwindende Momente (vanishing moments) aufweisen, denn dann sind die Koeffizienten $\nu_{m,n} = 0$.


\subsubsection{Vorgehen}
Suche FIR-Filter $M$ mit den Eigenschaft: 

\vspace{-0.4cm}

\begin{itemize}
	\itemsep-0.2cm
	\item $M(z) + M(-z)=2$
	\item $q$-fache Nullstelle bei $z=-1$
\end{itemize}

Vorgehen:

\vspace{-0.4cm}

\begin{itemize}
	\itemsep-0.2cm
	\item Zerlegen von $q$: $q = p + \tilde{p} \Rightarrow M(z)=2\cdot (\frac{z+1}{2})^p (\frac{z^{-1}+1}{2})^{\tilde{p}}M_0(z)$
	\item $M(-1)=0$ und $M(1) = 2$ muss $M_0(1)=1$ sein. $M_0(z) = A_0(z) \cdot \tilde{A}_0(z)$ mit $A_0(1)=\tilde{A}_0(1) = 1$
	\item $A(z) = \sqrt{2}(\frac{z^{-1}+1}{2})^p H_0(z)$ \qquad $\tilde{A}(z) = \sqrt{2}(\frac{z^{-1}+1}{2})^{\tilde{p}} \tilde{H}_0(z)$
\end{itemize}



\section{Scaling Function from Filter Coefficients}

\subsection{Variante 1}
Zu einem FIR-Filter $H(z)=\sum h_k z^{-k}$ muss eine dazugehörige Skalierungsfunktion $\varphi$ gefunden werden.
Wenn zu $H$ eine Skalierungsfunktion existiert, muss ihre FT $\hat{\varphi}$ die Gleichung $\hat{\varphi}(\xi)=P(\mathrm{e}^{j\pi\xi})\hat{\varphi}(\xi/2)$ erfüllen. Wobei $P:=\frac{\sqrt{2}}{2}H(z)$ gesetzt wird, dass der Faktor $\sqrt{2}$ wegfällt. Durch Iteration folgt daraus:
\[ \hat{\varphi}(\xi) = \prod_{k=0}^{\infty}P(\mathrm{e}^{j\pi\xi\cdot 2^{-k}}) \]
So kann $\hat{\varphi}$ nur durch $P$ ausgedrückt werden ($\varphi$ ist durch $\hat{\varphi}$ gegeben und $P$ durch $H$)!
Jetzt müssen noch folgenden Punkte bewiesen werden:

\vspace{-0.4cm}

\begin{enumerate}
	\itemsep-0.2cm
	\item Konvergenz von $\hat{\varphi}$
	\item $\hat{\varphi}$ eine $L^2$-Funktion ist (d.h. die Rücktransformation $\varphi$ existiert)
	\item $\varphi$ endlichen Träger aufweist
	\item $\varphi$ die MRA Axiome 1-3 erfüllt
\end{enumerate}
Probleme: Liegen im Beweis von (3.) und die Orthogonalitätsrelation (MRA Axiom 1).


\subsection{Variante 2}
Die Konstruktion verläuft vollständig im Zeitbereich. $\varphi^{[m]}$ wird rekursiv definiert. Es kann z.B. von $\varphi^{[0]}:=\varphi_{\text{Haar}}$ ausgegangen werden.
\[ 
	\varphi^{[m+1]}(t) := \sqrt{2} \sum_{k \in \mathbb{Z}} h_k \varphi^{[m]}(2t-k) 
	\qquad \qquad
	\varphi^{[m+1]} := \sum_{k \in \mathbb{Z}} h_k \varphi^{[m]}_{-1,k} 
\]

Die Orthogonalität wird induktive vom $\varphi^{[0]}$ auf die Glieder der Folge übertragen (geschieht auch mit dem MRA Axiom (3)).
Ist $h_k=0$ für $k<N_1$ und $k>N_2$, dann ist der Träger der Folge im Intervall $[N_1,N_2]$ für $m\rightarrow\infty$, wenn $\varphi^{[\infty]}$ punktweise konvergiert. Zudem ist dann $\varphi = \varphi^{[\infty]}$!\\

Problem: Nachweis der Konvergenz der Folge\\

Diskrete Version:
\[ 
	u_{m-1,n} = \sum_{k \in \mathbb{Z}} h_{n-2k}u_{m,k}
	\qquad \qquad
	y_{-m-1,n}=\sqrt{2} \sum_{k \in \mathbb{Z}} h_k y_{-m,n-2^mk}
\]
Beide Formeln ergeben das selbe Resultat wenn mit $y_{0,n}=u_{0,n}=\delta_{0,n}$ gestartet wird (bis auf Faktor $\sqrt{2}$).


\subsection{Regularität}
Mit der Regularität wird  die Stetigkeit, Differenzierbarkeit und mehrfache Differenzierbarkeit gemeint. (FT steckt dahinter)
Je schneller $\hat{\varphi}(\xi) \rightarrow 0$ geht (für $|\xi| \rightarrow \infty$), desto besser die Regularität von $\varphi$. Die Regularität von $\psi$ ist gleich der von $\varphi$, da $\psi$ ein Linearkombination von $\varphi$ ist. Die Regularität steigt mit der Ordnung von $p$.\\

Zusammenhang zwischen der Regularität von $\varphi$ und der Lokalisierung von $\hat{\varphi}$ ist allgemein durch folgenden Satz gegeben:\\
Wenn $|\hat{f}(\xi)| \leq \dfrac{C}{(1+|\xi|)^\delta}$ für alle $\xi$, mit Konstanten $C>0, 1\leq n+1 < \delta$ ($n$ ganz), dann ist $f$ mindestens $n$-mal differenzierbar und die $n$-te Ableitung von $f$ ist stetig.\\

Dieser Satz ist leider nicht umkehrbar und darum kann die Regularität besser sein als erwartet.
