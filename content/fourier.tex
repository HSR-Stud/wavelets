\section{Fourier Reihe}

Orthonormal Basis der Fourierreihe:
\[  
	e_k=\{\frac{1}{\sqrt{2 \pi}}, \frac{\cos(x)}{\sqrt{2 \pi}}, \frac{\sin(x)}{\sqrt{2 \pi}}, \frac{\cos(2x)}{\sqrt{2 \pi}}, \frac{\sin(2x)}{\sqrt{2 \pi}}, ... \} 
	\qquad \qquad
	f(x) = \sum_{k=0}^{\infty}\overbrace{\langle e_k|f \rangle}^{c_k} e_k
\]

Fourierreihe-Koeffizient berechnen: (wobei $\omega = \frac{2\pi}{T}$)
\[ 	a_k = \frac{2}{T} \langle \cos(k\omega t)|f \rangle 
	\qquad \qquad 
	b_k = \frac{2}{T} \langle \sin(k\omega t)|f \rangle 
	\qquad \qquad
	c_k = \frac{1}{T} \langle \e^{k\omega t}|f \rangle = \frac{1}{T} \int_{t}^{t+T}f(s) \e^{-jk\omega s} \, \mathrm{d}s
\]

\subsection{Properties}
\[  
	f(t) = \sum_{k \in \mathbb{Z}} p_k \e^{jk\omega t} \qquad \qquad g(t) = \sum_{k \in \mathbb{Z}} q_k \e^{jk\omega t} \qquad \qquad h(t) = \sum_{k \in \mathbb{Z}} c_k \e^{jk\omega t} \qquad h \in L^2([0,T])
\]

Linearität:\[ h=af+bg \Longrightarrow c_k=ap_k+bq_k \]
Translation:\[ h(t)=f(t+s) \Longrightarrow c_k=p_k\cdot \e^{jk\omega s} \]
Produkt:\[ h=f\cdot g \Longrightarrow c_k=\sum_{l\in\mathbb{Z}}p_l\cdot q_{k-l} \]
Faltung:\[ h(t)=(f\ast g)(t)=\int_{\tau}^{\tau + T}f(s)g(t-s) \, \mathrm{d}s \Longrightarrow c_k=p_k \cdot q_k \]
Derivative:\[ h=f' \Longrightarrow c_k=jk\omega p_k \]

Wenn folgendes Theorem erfüllt ist hat $h$ n kontinuierliche Ableitungen:
\[ |c_k|<\dfrac{c}{(1+|k|)^{1+n+\epsilon}} \qquad n \in \mathbb{N} \quad k \in \mathbb{Z} \]



\newpage
\section{Fourier Transformation}

\begin{center}
	\begin{tabular}{c|c|c}
		& periodisch & nicht-periodisch \\
		\hline
		diskret & $h_t \leftrightarrow \hat{h}_k$ & $c_k$, $h_t$ \\
		\hline
		kontinuierlich & $f(t+T)$, $\hat{h}(\xi)=\hat{h}(\xi + 1)$ & $f(t)\Leftrightarrow \hat{f}(\xi)$
	\end{tabular}
\end{center}

\subsection{Linear And Time Invariant System (LTI)}
Sei $x$ eine $T$ periodische Funktion, dann ist $H$ ein LTI System wenn folgendes gilt (S ist die Shift-Matrix um $T$):
\[ 	H\cdot S = S\cdot H \qquad \qquad y=H\cdot x 
	\qquad \qquad 
	\text{Bsp. wenn T=3 ist: } \left( \begin{array}{ccc} x_2=x_{-1} \\ x_0 \\ x_1 \end{array} \right) = S_3 \cdot \left( \begin{array}{ccc} x_0 \\ x_1 \\ x_2 \end{array} \right) \quad
	S_3=
	\left( \begin{array}{ccc}
	0 & 0 & 1 \\
	1 & 0 & 0 \\
	0 & 1 & 0 
	\end{array} \right) 
\]


\subsection{Finit Fourier Transformation}
Die Eigenvektoren und Eigenwerte plus ein Faktor von einer LTI Matrix $H$ stellen die Finite Fourier Transformation dar: (wobei $\omega = \frac{2\pi}{T}$)\\

Eigenwerte: \[ \lambda_k = \sum_{t=0}^{T-1} h_t \e^{-jk\omega t} \]
Finit Fourier Transformation: \[ \hat{h}_k = c_T \cdot \lambda_k = c_T \cdot \sum_{t=0}^{T-1} h_t \e^{-jk\omega t} \]
Inverse Finit Fourier Transformation: \[ h_t = \frac{1}{c_T T} \cdot \sum_{k=0}^{T-1}\hat{h}_k \e^{+jk\omega t} \]
Wobei $c_T = \frac{1}{T}$. Mehr Symmetrie kann durch die Wahl von $c_T = \frac{1}{\sqrt{T}}$ erreicht werden. Dann gilt $||h||^2=||\hat{h}||^2$

(FFT rein nehmen ???)

\subsection{Fourierreihe, Laurentreihe, Z-Transformation}

Transferfunction: \[ H(z)=\sum h_k \cdot z^{-k} \qquad \text{(Z-Transformation)} \qquad \qquad z_k=\e^{jk\omega} \qquad z_k^0=z_k^T=1 \]
Frequenzgang: \[ \hat{h}(\xi) = H(\e^{2\pi j \xi})=\sum h_k \e^{-2\pi jk \xi} \qquad \text{Aplitudengang: } |\hat{h}(\xi)| \qquad \text{Phasengang: } \arg|\hat{h}(\xi)| \]

(Die DFT ist nur die inverse Fourierreihe mit Periode $T=1$ und einem Signum.)

\subsection{Fouriertransformation on $\mathbf{L^2(\mathbb{R})}$}

Fouriertransform (FT): \[ \hat{f}(\xi) = \int_{-\infty}^{\infty}f(t) \e^{-j 2\pi \xi t} \, \mathrm{d}t \qquad \qquad \xi = \frac{k}{T} \text{ wobei } (T \rightarrow \infty) \]

Inverse Fouriertransform (IFT): \[ f(t) = \int_{-\infty}^{\infty}\hat{f}(\xi) \e^{+j 2\pi \xi t} \, \mathrm{d}\xi \]

Plancherel Identity: \[ ||f||^2 = ||\hat{f}||^2 \]

\subsubsection{Properties}
Linearität: \[ h = af + bg \FT \hat{h} = a\hat{f + b\hat{g}} \]
Translation: \[ h(t) = f(t+s) \FT \hat{h}(\xi) = \hat{f}(\xi) \cdot \e^{j2\pi\xi s} \]
Produkt: \[ h = f \cdot g \FT \hat{h} = (\hat{f}\ast\hat{g})(\xi) = \int \hat{f}(\xi) \hat{g}(\xi - s) \,\mathrm{d}s \]
Faltung: \[ h = (f\ast g)(t) \FT \hat{h} = \hat{f} \cdot \hat{g} \]
Derivative: \[ h = f' \FT \hat{h} = j2\pi\xi\cdot\hat{f}(\xi) \]
Scaling: \[ h(t) = f(at) \FT \hat{h}(\xi) = \frac{1}{|a|} \cdot \hat{f}(\frac{\xi}{a}) \]

Unschärefe, Sampling Theorem und Aliasing rein nehmen???
