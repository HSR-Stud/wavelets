\section{Faltung}
Faltung:
\begin{equation}
  \nonumber
  y(t) = f(t) \ast g(t) = \int\limits_{-\infty}^{\infty}f(u) \cdot g(t-u) du
\end{equation}

Diskret:
\begin{equation}
  \nonumber
  y(i)=f(i) \ast g(i)=\sum\limits_{k=-\infty}^{\infty}f(k)\cdot g(i-k)
\end{equation}

\begin{tabular}{p{9cm}p{9cm}}
  Interpretation: & Geltende Rechengesetze:\\
  \begin{enumerate}
    \item $g(t)$ an der Y-Achse Spiegeln
	  \item Verschiebung um $t$ nach rechts
	  \item Durch Multiplikation der beiden Signale entstehende Fläche berechnen
  \end{enumerate}
&
  \begin{itemize}
    \item Kommutativ (Vertauschen) $g \ast f = f \ast g $
	  \item Distributiv (Ausklammern, Ausmultiplizieren) $g \ast(f \ast h) = g \ast f \ast h$
	  \item Assoziativ (Reihenfolge von Klammern) $g \ast(f \ast h) = (g \ast f) \ast h$
	  \item Faltung mit Dirac $\delta(t-t_0) \ast f(t) = f(t-t_0)$
  \end{itemize}
\end{tabular}
	
	
	
