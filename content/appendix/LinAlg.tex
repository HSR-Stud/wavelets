\section{Lineare Algebra}

\subsection{Vektoren}

\subsubsection{Betrag von Vektoren}
\[ 
	|\vec{a}| = \sqrt{\vec{a} \cdot \vec{a}} = ||\vec{a}|| =\sqrt{a_1^2 + a_2^2 + \ldots + a_n^2}
\]

	
\subsubsection{Skalarprodukt}
\[ 
	\vec{a} \cdot \vec{b} = a_1 \cdot b_1 + a_2 \cdot b_2 + \ldots +a_n \cdot b_n = |\vec{a}| \cdot |\vec{b}| \cdot \cos(\varphi)
\]
			
Projektion $\vec{p}$ vom Vektor $\vec{a}$ auf den Einheitsvektor $\vec{e}$ (wobei $|\vec{e}|=1$): \hspace{1cm} $\vec{p}=(\vec{a} \cdot \vec{e}) \cdot \vec{e}$


\subsubsection{Vektorprodukt}
\[
	\vec{a} \times \vec{b} = 
	\begin{pmatrix}a_1\\a_2\\a_3 \end{pmatrix} \times 
	\begin{pmatrix}b_1\\b_2\\b_3 \end{pmatrix} = 
	\begin{pmatrix}
	a_2 b_3 - a_3 b_2 \\ \mathbf{a_3 b_1 -  a_1 b_3} \\ a_1 b_2 - a_2 b_1 
	\end{pmatrix} \quad \quad 
	|\vec{a} \times \vec{b}|=|\vec{a}| \cdot |\vec{b}| \cdot
	\sin(\varphi)
\]

Der Betrag von $|\vec{a} \times \vec{b}|$ entspricht der Fläche des
Parallelogramms das von $\vec{a}$ und $\vec{b}$ aufgespannt wird. Der Vektor  $\vec{a} \times \vec{b}$ steht senkrecht auf den Vektoren $\vec{a}$ und $\vec{b}$!

	
\subsection{Matrizen}

\subsubsection{Matrizenprodukt}
Matrizenprodukt (Skalarprodukt) einer Matrix A und einer Matrix B ergiebt eine
Matrix C, deren Elemente die Skalarprodukte der Zeilenvektoren von A mit den
Spaltenvektoren von B sind.

\[
	\begin{pmatrix}
	a_{11} & \ldots & a_{1n} \\
	\vdots & A & \vdots \\
	a_{n1} & \ldots & a_{nn}
	\end{pmatrix} \cdot \begin{pmatrix} 
	b_{11} & \ldots & b_{1n} \\
	\vdots & B & \vdots \\
	b_{n1} & \ldots & b_{nn}
	\end{pmatrix} = \begin{pmatrix}
	\sum\limits_{i=1}^n a_{1i}b_{i1} & \ldots & \sum\limits_{i=1}^n a_{1i}b_{in}
	\\ \vdots & C & \vdots \\
	\sum\limits_{i=1}^n a_{ni}b_{i1} & \ldots & \sum\limits_{i=1}^n a_{ni}b_{in}
	\end{pmatrix}
\]


\subsubsection{Rang}
Der Rang einer Matrix gibt an, wie viele Zeilen linear unabhängig sind.


\subsubsection{Spur}
Die Spur einer Matrix $A$ wird durch die Summe der Diagonalelemente berechnet.

\[
	\mathrm{Spur}(A)=\mathrm{tr}(A)=\sum\limits_{i=1}^n a_{ii}
\]

Rechenregeln:\\
- $\mathrm{Spur}(A)=\mathrm{Spur}(A^t)$\\
- $\mathrm{Spur}(AB)=\mathrm{Spur}(BA)$\\
- $\mathrm{Spur}(ABC)=\mathrm{Spur}(CBA)=\mathrm{Spur}(BCA)$\\

Anwendung: $\mathrm{Spur}(A)=1+2 \cos(\alpha)$

\subsubsection{Transponierte Matrix}
Transponierte Matrix $A^t$ heisst alle Spalten und Zeilen
der Matrix $A$ vertauschen.

\[
	A=  
	\begin{pmatrix} 
		a_{11} & a_{12} & \cdots & a_{1n} \\
		a_{21} & a_{22} & \cdots & a_{2n} \\
		\vdots & \vdots & \ddots & \vdots \\
		a_{m1} & a_{m2} & \cdots & a_{mn} \\
	\end{pmatrix}
	\qquad \qquad 
	A^{\mathrm{T}} = 
	\begin{pmatrix} 
		a_{11} & a_{21} & \cdots & a_{m1} \\
		a_{12} & a_{22} & \cdots & a_{2n} \\
		\vdots & \vdots & \ddots & \vdots \\
		a_{1n} & a_{2n} & \cdots & a_{mn} \\
	\end{pmatrix}
\]

Rechenregeln:\\
- $(A+B)^t=A^t+B^t$\\
- $(A \cdot B)^t=B^t \cdot A^t$


\subsubsection{Orthogonale Matrix}
Eigenschaften:\\
- Quadratisch\\
- Zeilenvektoren sind orthonormal, Spaltenvektoren sind orthonormal\\
- Multiplikation von 2 orthogonalen Matrizen gibt eine orthogonale
Matrix\\
- $A^t=A^{-1}$ und $A^t \cdot A = I$(Einheitsmatrix)\\
- Determinante hat immer Betrag $1$\\
- $\det(O)=+1$: die orthogonale Matrix $O$ enthält alle Drehungen\\
- $\det(O)=-1$: die orthogonale Matrix $O$ enthält alle Drehspiegelungen 


\subsubsection{Determinante (Quadratische Matrizen)}
\textbf{Berechnung:} \\
Die Determinante ist das Produkt der Pivoelemente $p$ (mit
Gauss ermittelbar).  $\det \begin{pmatrix}p_1 & \ldots & \ast \\
\vdots & p_i & \vdots \\ \ast & \ldots & p_n \end{pmatrix} =
\sum\limits_{i=1}^{n}p_i$\\

2D Matrix (Fisch): \hspace{1cm} $\det \begin{pmatrix}a & b \\ c
& d \end{pmatrix} = a d - b c$\\

\textbf{Rechenregeln:}\\
- $\det(A \cdot B) = \det(A) \cdot \det(B)$\\
- $\det(A^{-1})=\det(A)^{-1}=\frac{1}{\det(A)}$\\
- $\det(A^t)=\det(A)$ wenn $A$ Quadratische \\

\textbf{Merksätze:}\\
- Matrix $A$ linear abhängig $\Rightarrow \det(A)=0$\\
- Vertauschen von 2 Zeilen $\Rightarrow (-1)\cdot \det(A)$\\
- Determinante einer 2D Matrix entspricht der Fläche des aufgespannten
Parallelogramm\\
- Determinante einer 3D Matrix entspricht dem Volumen des aufgespannten
Parallelepipeds\\
- Determinante einer orthagonalen Matrix hat immer Betrag $1$

\subsubsection{Matrixinversion}
	$A^{-1} = \begin{pmatrix}
	a & b \\ c & d \\
	\end{pmatrix}^{-1} =
	\frac{1}{\det(A)} \begin{pmatrix}
	d & -b \\ -c & a \\
	\end{pmatrix}  =
	\frac{1}{ad-bc} \begin{pmatrix}
	d & -b \\ -c & a \\
	\end{pmatrix}\quad$\\
	$A^{-1} = \begin{pmatrix}
	a & b & c\\ d & e & f \\ g & h & i \\
	\end{pmatrix}^{-1} =
	\frac{1}{\det(A)} \begin{pmatrix}
	ei - fh & ch - bi & bf - ce \\
	fg - di & ai - cg & cd - af \\
	dh - eg & bg - ah & ae - bd
	\end{pmatrix}$\\

\subsubsection{Cramersche Regel}
Ausgangslage: $\qquad A \cdot x = b \qquad \det(A)\neq 0$\\
Lösung $x_i$:
\[
	x_i=\frac{\det(A_i)}{\det(A)} \qquad \qquad A_i:\text{ $i$-te Spalte von $A$  durch rechte Seite  $b$ ersetzen!}
\]


\subsection{Eigenvektoren und Eigenwerte}
Ein Eigenvektor $v$ einer Abbildung $A$ (Matrix) ist ein vom Nullvektor
verschiedener Vektor, dessen Richtung durch die Abbildung nicht verändert wird. Der Vektor wird nur
gestreckt oder gestaucht. Dieser Faktor wird Eigenwert $\lambda$ genannt.
\[
	A v = \lambda v \quad \Leftrightarrow \quad (A-\lambda I)v=0
\]

\textbf{Charakteristisches Polynom:}\\
\[
	\chi_A(\lambda)=\det(A-\lambda I)
\]

\textbf{Eigenwerte:}\\
Die Nullstelle (da $A-\lambda I$ Singulär) des charakteristischen Polynom sind
die Eigenwerte. 
\[
	\chi_A(\lambda)=\det(A-\lambda I)=0
\]

\textbf{Eigenvektoren:}\\
Um die Eigenvektoren zu erhalten muss das folgende Gleichungssystem für jeden
Eigenwert gelöst werden.
\[ 
	(A-\lambda I)v=0 \qquad \qquad (v \text{ auf länge 1 normieren})
\]

\subsection{Spezielle Matrizen}
Einheitsmatrix: 
$I= \begin{pmatrix} 
1 & 0 & 0 \\
0 & 1 & 0 \\
0 & 0 & 1
\end{pmatrix}$

\vspace{0.5cm}

Drehmatrizen:
\\
$\quad$
X-Achse
$\begin{pmatrix} 
	1 & 0 & 0 \\
	0 & \cos(\alpha) & -\sin(\alpha) \\
	0 & \sin(\alpha) & \cos(\alpha)
\end{pmatrix}$
$\quad$
Y-Achse
$\begin{pmatrix} 
	\cos(\alpha) & 0 & \sin(\alpha) \\
	0 & 1 & 0 \\
	-\sin(\alpha) & 0 & \cos(\alpha)
\end{pmatrix}$
$\quad$
Z-Achse
$\begin{pmatrix} 
	\cos(\alpha) & -\sin(\alpha) & 0 \\
	\sin(\alpha) & \cos(\alpha) & 0 \\
	0 & 0 & 1
\end{pmatrix}$
